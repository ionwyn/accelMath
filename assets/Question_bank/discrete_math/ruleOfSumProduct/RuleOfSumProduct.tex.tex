\documentclass{article}
\usepackage[utf8]{inputenc}
\usepackage{enumitem}
\newcommand\tab[1][1cm]{\hspace*{#1}}
\usepackage{amsmath}

\title{accelMath Questions: Discrete Math 1}
\author{ionwynsean }
\date{March 2017}

\begin{document}

\maketitle

\section{Rule of Sum and Product Warmup}

\begin{enumerate}
    \item A palindrome is a number that reads the same forwards and backwards (like 747). Which of the following numbers is a palindrome?

    
        \begin{itemize}
            \item 100
            \item 101
            \item 102
            \item All of the above
        \end{itemize}
        
    \item How many 3-digit palindromes are there? (Note: The first digit of a 3-digit number must be nonzero.)
    
    \begin{itemize}
        \item Fewer than 100
        \item Exactly 100
        \item More than 100
    \end{itemize}
    
    \item If $A$ is the number of 3-digit palindromes and $B$ is the number of 4-digit palindromes, which is greater?
    
    \begin{itemize}
        \item $A$
        \item $B$
        \item $A$ and $B$ are equal
    \end{itemize}
    
    Solution:
    First let's compute LaTeX: \(A\), the number of 3-digit palindromes. Imagine constructing a 3-digit palindrome by choosing the digits sequentially, starting from the left. There are 9 choices for the first (leftmost) digit, since it can be any digit other than 0. After the first digit has been chosen, there are 10 choices for the second digit (regardless of the choice for the first digit). In order for the number to be a palindrome, the last digit must be the same as the first digit, so after the first digit has been chosen, there is only 1 choice for the last digit. The total number of ways to choose the digits is the product of these numbers:
    LaTeX: \[A = 9 \times 10 \times 1 = 90\]
    Now let's compute LaTeX: \(B\), the number of 4-digit palindromes. Again, imagine constructing a 4-digit palindrome by choosing the digits sequentially, starting from the left. There are 9 choices for the first (leftmost) digit, since it can be any digit other than 0. After the first digit has been chosen, there are 10 choices for the second digit (regardless of the choice for the first digit). In order for the number to be a palindrome, the third digit must be the same as the second digit and the last digit must be the same as the first digit, so after the first and second digits have been chosen, there is only 1 choice for the third digit and 1 choice for the last digit. The total number of ways to choose the digits is the product of these numbers:
    LaTeX: \[B = 9 \times 10 \times 1 \times 1 = 90\]
    Thus, LaTeX: \(A = B\).
    
    \item How many 3-digit palindromes are there where the middle digit is larger than the outer digits (like 121)?
    
    \begin{itemize}
        \item Fewer than 45
        \item Exactly 45
        \item More than 45
    \end{itemize}
    
    \item How many 3-digit palindromes are there where the middle digit is the same as the leading digit?
    
    \begin{itemize}
        \item Fewer than 10
        \item Exactly 10
        \item More than 10
    \end{itemize}

\end{enumerate}

\section{Rule of Sum and Product}

Lesson:
The rule of sum is a basic counting approach in combinatorics. A basic statement of the rule is that if there are \( n\) choices for one action, and \( m\) choices for another action and the two actions cannot be done at the same time, then there are \( n+m\) ways to choose one of these actions.

\begin{enumerate}
    \item A university has three parking lots, call it A, B, and C.  Lot A has 80 spaces, lot B has 90 spaces, and lot C has 310 spaces.  If a car comes to the university, how many choices are there for the car to park in the parking space?
    
    \begin{itemize}
        \item 480
        \item 310
        \item 300
        \item uncountable
    \end{itemize}
    
    Answer: 480
    
    \item How many triples of positive integers \( (a, b, c) \) are there such that
    
    \[\frac{a^2 + b^2}{c}, \quad <4a\le 5, \quad b\le 5, \quad c\le 5? \]
    
    \begin{itemize}
        \item 19
        \item 24
        \item 29
        \item 34
        \item 39
    \end{itemize}
    
    Answer: 29
    
    Since all of  \(a, b\) and  \(c\) are positive integers between  \(1\) and  \(5,\) inclusive, we check each of the options as follows:

If  \( c = 5, \) then  \((a, b)=(4,1)\) or  \((3,3)\) or  \((3,2)\) or  \((3,1)\) or  \((2,3)\) or  \((2,2)\) or  \((2, 1)\) or  \((1,4)\) or  \((1, 3)\) or  \((1, 2)\) or  \((1, 1),\) which gives  \(11\) solutions.

If  \( c = 4, \) then  \((a, b)=(3,2)\) or  \((3,1)\) or  \((2,3)\) or  \((2,2)\) or  \((2, 1)\) or  \((1, 3)\) or  \((1, 2)\) or  \((1, 1),\) which gives  \(8\) solutions.

If  \( c = 3, \) then  \((a, b)=(3,1)\) or  \((2,2)\) or  \((2, 1)\) or  \((1, 3)\) or  \((1, 2)\) or  \((1, 1),\) which gives  \(6\) solutions.

If  \( c = 2, \) then  \((a, b)=(2, 1)\) or  \((1, 2)\) or  \((1, 1),\) which gives  \(3\) solutions.

If  \( c = 1, \) then  \((a, b)=(1, 1),\) which gives  \(1\) solution.

Hence, there are  \(11+8+6+3+1=29\) solutions altogether.

    \item While training for the Olympics, Sebastian as to swim, then cycle, and then run.  To practice, he can either 200, 300, 400, or 500 meters, then cycle for 5, 8, 13, or 21 kilometers, and finally run for either 2 or 3 kilometers.  How many possible choices of training does Sebastian have?
    
    \begin{itemize}
        \item 1
        \item 3
        \item 10
        \item 32
    \end{itemize}
    
    Answer:
    32
    
    Since Trina will be doing each activity, we use the rule of product. So Sebastian will have \(4 \times 4 \times 2 = 32\) possible choices for her training.
    
    \item In Alice's closet, there are $4$ different pants, $9$ different shirts and $n$ different jackets. If the number of combinations of pants, shirts and jackets Alice can choose is $252$ how many different jackets does she have?
    
    \begin{itemize}
        \item 7
        \item 9
        \item 13
        \item 239
    \end{itemize}
    
    Answer: 7
    For each of the  \(n\) different jackets, there are  \(4\) different pants and  \(9\) different shirts to choose from. Thus, the total number of combinations is  \(4 \times 9 \times n,\) implying

 \[4 \times 9 \times n & = 252 \]
 \[n = 7 \]
 

Thus, Alice has  \(7\) different jackets.

    \item Find the number of $2$ digit numbers whose digit sum is a perfect square.
    
    \begin{itemize}
        \item 5
        \item 14
        \item 20
        \item 17
    \end{itemize}
    
    Answer: 17
    
    Let  \(n=10a+b\). We have that  \(1<=a<=9\) and  \(0<=b<=9\), so  \(0<=a+b<=18\). The perfect squares for that range is  \(1,4,9 \) and  \(16\). For the case  \(1\),  \(10\) is unique because you can't raise  \(b\) and lower  \(a\) and  \(a+b\) remain as  \(1\). For the case  \(4\) the higher value is  \(40\), so we have  \(4\) possibilities because  \(1<=a<=4\). For the case  \(9\),  \(90\) is the higher value,  \(1<=a<=9\) so we have  \(9\) possibilities. Finally, for the case  \(16\),  \(97\) is the higher value, as we lower  \(a\) and raise  \(b\) we will have only  \(3\) possibilities because  \(b\) already started at  \(7\). Summing all possibilities:  \(1+4+9+3=17\).
    
    \item If you take a  \(3 \times 3 \times 3\) Rubik cube and break it up into the  \( 3^3 \) cubes, how many cubes would have stickers on exactly  \(2\) faces?
    
    \begin{itemize}
        \item 6
        \item 8
        \item 12
        \item 26
    \end{itemize}
    
    Answer: 12
    
    Typical Rubik's cube ($3 \times 3 \times 3$) has 8 corners and 12 sides.  There is only one cube in between the two corner cubes at each side. This yields the number of cubes having sticker on two side be  \(12\times1 =\boxed{12}\).
    
\end{enumerate}

\end{document}
