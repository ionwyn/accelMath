\documentclass{article}
\usepackage[utf8]{inputenc}

\title{Discrete Math Binomial Theorem}
\author{ionwynsean }
\date{March 2017}

\begin{document}

\maketitle

\section{Introduction}
The binomial theorem is a result about expanding the powers of binomials, or sums of two terms. The coefficients of the terms in the expansion are the binomial coefficients  \( {N}\choose{k} \). The theorem and its generalizations can be used to prove results and solve problems in combinatorics, algebra, calculus, and many other areas of mathematics.

\section{Questions}

\begin{enumerate}
    \item 5 people became friends at a Summer Camp. On the last day, they shook each other's hands to say farewell. How many handshakes were there?
    
    Answer: 10
    
    Solution 1: Let's count it slowly, by per person. 
    The first person has 4 other hands to shake. Now let's ignore the first person. 
    The second person has 3 other hands to shake (excluding the first person). Now let's ignore the second person. 
    The third person has 2 other hands to shake (excluding the first and second person). Now let's ignore the third person. 
    The fourth person has 1 other hands to shake (excluding the first, second and third person). This is all that remains.
    
    Hence, there are  \( 4 + 3 + 2 + 1 = 10 \) handshakes that occurred.
    
    Solution 2: We can draw a graph to represent this. There are 5 vertices representing each of the students. Between each vertex, we draw an edge to represent a handshake. We can count that there are 10 edges, hence there are 10 handshakes that occurred.


    \item Properties of Binomial Coefficients:
    
    What positive integer  \(n\) satisfies
     \[{{n+2} \choose 7}=41 \cdot {n \choose 5}?\]
 
 
    \begin{itemize}
        \item 108
        \item 41
        \item 40
        \item 0
    \end{itemize}
    Answer: 40
    
    Observe that the equation can be rewritten as

     \[{{n+2} \choose 7} =41 \cdot {n \choose 5} \\ \frac{(n+2)(n+1)n(n-1)}{7!} =41 \times \frac{n(n-1)}{5!}.\]
    
    Since  \(n\) is positive, dividing both sides of this by  \(n(n-1)\) gives
     \[(n+2)(n+1) =42 \cdot 41.\]

    This is satisfied for  \(n=40\).
    
    \item Binomial Theorem Expansions
    
    For what value of  \(N\), is the following an algebraic identity:

     \[ (x+y)^N = x^4 + 4 x^3 y + 6 x^2y^2 + 4 xy^3 + y^4? \]
     
     
     \begin{itemize}
         \item 1
         \item 2
         \item 3
         \item 4
     \end{itemize}
     
     Answer: 4
     
     By the binomial theorem, we know that

     \[ (x+y)^4 = = x^4 + 4 x^3 y + 6 x^2y^2 + 4 xy^3 + y^4. \]
     
     \item Binomial Theorem challenge:
     
      \[
      \sum_{ r=0 }^{ n } { (-1 )}^{ r } {{n}\choose{r}}^{-1} \]
      
      If  \(n\) is an an odd positive integer, find the value of this sum.
     
     \begin{itemize}
         \item -1
         \item 0
         \item 1
         \item Infinity
     \end{itemize}    
     
     Answer: 0
     
     Recall that  \({{n}\choose{r}} = {{n}\choose{n - r}}.\) Since  \(n\) is odd,  \(n-r\) has the opposite parity as  \(r,\) so

     \[{(-1)^r} {{{n}\choose{r}}^{-1}} = -(-1)^{n-r}{{n}\choose{n-r}}^{-1}.\]
    
    Thus, when we pair the terms in this way, each pair sums to 0, so the summation is 0.
    
    \item Binomial Theorem Challenge 2:
    
     \[ \large \sum_{k=0}^n \left[ (-1)^k {{n}\choose{k}} (n-k)^n \right] = \ ? \]
     
     \begin{itemize}
         \item n!
         \item 1
         \item (n-1)!
         \item 0
     \end{itemize}
     
     Answer: n!
    
    In a bijective, number of elements in domain = number of elements of co-domain = (n) and range is equal to co-domain.
    
    In a bijective, No. of one-one functions is equal to number of onto functions.
    
    No.of. onto functions =  \(\sum_{k=0}^n(-1)^k{{n}\choose{n-k}}^m\).
    
    Here, m = n.
    
    So,  \(\sum_{k=0}^n(-1)^k{{n}\choose{n-k}}^n\) = number of one-one functions from n to n = n!.

    \item Binomial Theorem Challenge 3:
    
    Find the number of odd coefficients in the expansion of  \( (a + b)^{2015} \).
    
    \begin{itemize}
        \item 512
        \item 1024
        \item 2048
        \item 2047
    \end{itemize}
    
    Answer: 1024
    
    We will make use of a consequence of Lucas' Theorem, whereby a binomial coefficient  \({{m}\choose{n}}\) is odd if and only if none of the digits in the binary expansion of  \(n\) is greater than the corresponding digit in the binary expansion of  \(m.\)

    In this case we have  \(m = 2015,\) which has a binary of 11111011111.
    
    So all those binomial coefficients  \({{2015}\choose{n}}\) in the expansion of  \((a + b)^{2015}\) for which  \(n\) does not have a  \(1\) in the  \(6\)th from the right digit in their binary expansion will be odd. 
    
    Since all numbers from  \(2016\) to  \(2047\) do have a  \(1\) in this position, we can then conclude that
    the number of values \(n\) from \(0\) to \(2015\) inclusive that do not have a  \(1\) in this position will be  \(\frac{2048}{2} = 1024.\)
    
\end{enumerate}

\end{document}
